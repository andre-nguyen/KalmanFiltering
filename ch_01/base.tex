\documentclass{article}

\usepackage{fancyhdr}
\usepackage{extramarks}
\usepackage{amsmath}
\usepackage{amsthm}
\usepackage{amsfonts}
\usepackage{tikz}
\usepackage[plain]{algorithm}
\usepackage{algpseudocode}

\usetikzlibrary{automata,positioning}

%
% Basic Document Settings
%

\topmargin=-0.45in
\evensidemargin=0in
\oddsidemargin=0in
\textwidth=6.5in
\textheight=9.0in
\headsep=0.25in

\linespread{1.1}

\pagestyle{fancy}
\lhead{\hmwkAuthorName}
\chead{\hmwkClass\ : \hmwkTitle}
\rhead{\firstxmark}
\lfoot{\lastxmark}
\cfoot{\thepage}

\renewcommand\headrulewidth{0.4pt}
\renewcommand\footrulewidth{0.4pt}

\setlength\parindent{0pt}

%
% Create Problem Sections
%

\newcommand{\enterProblemHeader}[1]{
    \nobreak\extramarks{}{Problem \arabic{#1} continued on next page\ldots}\nobreak{}
    \nobreak\extramarks{Problem \arabic{#1} (continued)}{Problem \arabic{#1} continued on next page\ldots}\nobreak{}
}

\newcommand{\exitProblemHeader}[1]{
    \nobreak\extramarks{Problem \arabic{#1} (continued)}{Problem \arabic{#1} continued on next page\ldots}\nobreak{}
    \stepcounter{#1}
    \nobreak\extramarks{Problem \arabic{#1}}{}\nobreak{}
}

\setcounter{secnumdepth}{0}
\newcounter{partCounter}
\newcounter{homeworkProblemCounter}
\setcounter{homeworkProblemCounter}{1}
\nobreak\extramarks{Problem \arabic{homeworkProblemCounter}}{}\nobreak{}

%
% Homework Problem Environment
%
% This environment takes an optional argument. When given, it will adjust the
% problem counter. This is useful for when the problems given for your
% assignment aren't sequential. See the last 3 problems of this template for an
% example.
%
\newenvironment{homeworkProblem}[1][-1]{
    \ifnum#1>0
        \setcounter{homeworkProblemCounter}{#1}
    \fi
    \section{Problem \arabic{homeworkProblemCounter}}
    \setcounter{partCounter}{1}
    \enterProblemHeader{homeworkProblemCounter}
}{
    \exitProblemHeader{homeworkProblemCounter}
}

%
% Homework Details
%   - Title
%   - Due date
%   - Class
%   - Section/Time
%   - Instructor
%   - Author
%

\newcommand{\hmwkTitle}{Chapter 1 problems}
\newcommand{\hmwkDueDate}{February 12, 2014}
\newcommand{\hmwkClass}{Kaman Filtering}
\newcommand{\hmwkClassTime}{Section A}
\newcommand{\hmwkClassInstructor}{Professor Isaac Newton}
\newcommand{\hmwkAuthorName}{Andre Phu-Van Nguyen}

%
% Title Page
%

\title{
    \vspace{2in}
    \textmd{\textbf{\hmwkClass:\ \hmwkTitle}}\\
    %\normalsize\vspace{0.1in}\small{Due\ on\ \hmwkDueDate\ at 3:10pm}\\
    %\vspace{0.1in}\large{\textit{\hmwkClassInstructor\ \hmwkClassTime}}
    \vspace{3in}
}

\author{\textbf{\hmwkAuthorName}}
\date{}

\renewcommand{\part}[1]{\textbf{\large Part \Alph{partCounter}}\stepcounter{partCounter}\\}

%
% Various Helper Commands
%

% Useful for algorithms
\newcommand{\alg}[1]{\textsc{\bfseries \footnotesize #1}}

% For derivatives
\newcommand{\deriv}[1]{\frac{\mathrm{d}}{\mathrm{d}x} (#1)}

% For partial derivatives
\newcommand{\pderiv}[2]{\frac{\partial}{\partial #1} (#2)}

% Integral dx
\newcommand{\dx}{\mathrm{d}x}

\DeclareMathOperator{\tr}{tr}

% Alias for the Solution section header
\newcommand{\solution}{\textbf{\large Solution}}

% Probability commands: Expectation, Variance, Covariance, Bias
\newcommand{\E}{\mathrm{E}}
\newcommand{\Var}{\mathrm{Var}}
\newcommand{\Cov}{\mathrm{Cov}}
\newcommand{\Bias}{\mathrm{Bias}}

\begin{document}

\maketitle

\pagebreak

% !TEX root = base.tex

\begin{homeworkProblem}
  Prove Lemma 1.4 \textit{If $A$ and $B$ are $n \times n$ matrices, then}
\begin{align}
  \tr A^\top      &= \tr A\\
  \tr(A + B)      &= \tr A + \tr B\\
  \tr(\lambda A)  &= \lambda \tr A\\
  \tr(A^\top A)   &= \sum_{i=1}^n \sum_{j=1}^m a^2_{ij}
\end{align}

\solution

By definition the trace of a matrix is the sum of the diagonal elements
\[
  \tr A = \sum_{i=1}^n a_{ii}
\]

So for $n = 3$ then we have
\[
  A = \begin{bmatrix}
  a_{11} & a_{12} & a_{13}\\
  a_{21} & a_{22} & a_{23}\\
  a_{31} & a_{32} & a_{33}
\end{bmatrix}, \lambda A = \begin{bmatrix}
\lambda a_{11} & \lambda a_{12} & \lambda a_{13}\\
\lambda a_{21} & \lambda a_{22} & \lambda a_{23}\\
\lambda a_{31} & \lambda a_{32} & \lambda a_{33}
\end{bmatrix},\ A^\top = \begin{bmatrix}
a_{11} & a_{21} & a_{31}\\
a_{12} & a_{22} & a_{32}\\
a_{13} & a_{23} & a_{33}
\end{bmatrix}, \ B = \begin{bmatrix}
b_{11} & b_{12} & b_{13}\\
b_{21} & b_{22} & b_{23}\\
b_{31} & b_{32} & b_{33}
\end{bmatrix}
\]
\[
A + B = \begin{bmatrix}
a_{11} + b_{11} & a_{12} + b_{12} & a_{13} + b_{13}\\
a_{21} + b_{21} & a_{22} + b_{22} & a_{23} + b_{23}\\
a_{31} + b_{31} & a_{32} + b_{32} & a_{33} + b_{33}
\end{bmatrix}
\]

The proof of (1) is trivial:
\[
  \tr A = a_{11} + a_{22} + a_{33} = \tr A^\top
\]

The proof of (2) is also trivial:
\[
  \tr (A+B) = a_{11} + b_{11} + a_{22} + b_{22} + a_{33} + b_{33}
\]
\[
  \tr A + \tr B = a_{11} + a_{22} + a_{33} + b_{11} + b_{22} + b_{33}
\]

The proof of (3) is also also trivial:
\[
  \tr(\lambda A) = \lambda a_{11} + \lambda a_{22} + \lambda a_{33} = \lambda (a_{11} + a_{22} + a_{33})
  = \lambda \tr A
\]

The proof of (4) is

\[
  A =
    \begin{bmatrix}
      a & b\\
      c & d
    \end{bmatrix},
  A^\top =
    \begin{bmatrix}
      a & c\\
      b & d
    \end{bmatrix}
\]
\[
  A^\top A =
  \begin{bmatrix}
    a^2+b^2 & ac+bd\\
    ac+bd & c^2+d^2
  \end{bmatrix}
\]
\[
  \tr ( A^\top A ) = a^2 + b^2 + c^2 + d^2
\]

\end{homeworkProblem}

\pagebreak

\begin{homeworkProblem}
  Prove lemma 1.6 \textit{Let $A$ be an $n \times n$ matrix. Then}
  \begin{align}
      \tr A \leq (n \tr AA^\top)^{1/2}
  \end{align}

  \solution
\[
  \tr A = \sum_{i=1}^n a_{ii}
\]
\[
  \tr(A^\top A)   &= \sum_{i=1}^n \sum_{j=1}^m a^2_{ij}
\]

\[
  \sum_{i=1}^n a_{ii} &\leq (n \sum_{i=1}^n \sum_{j=1}^n a^2_{ij})^{1/2}
\]
Square all the terms
\[
  (\sum_{i=1}^n a_{ii})^2 &\leq n \sum_{i=1}^n \sum_{j=1}^n a^2_{ij}
\]
Square of sums $\leq$ sum of squares
\[
  (\sum_{i=1}^n a_{ii})^2 &\leq n \sum a_{ii}^2 \leq n \sum_{i=1}^n \sum_{j=1}^n a^2_{ij}
\]
\end{homeworkProblem}

\pagebreak

\begin{homeworkProblem}
  Give an example of two matrices $A$ and $B$ such that $A \geq B \geq 0$ but for which the inequality $AA^\top \geq BB^\top$ is not satisfied.

  \solution


\end{homeworkProblem}


\end{document}
